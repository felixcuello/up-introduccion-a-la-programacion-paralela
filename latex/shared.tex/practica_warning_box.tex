\begin{tcolorbox}[colback=yellow,colframe=red!75!black,arc=0pt,outer arc=0pt]
  \textbf{ATENCIÓN CON LAS PRÁCTICAS} \\

En programación, la práctica es un componente fundamental del aprendizaje. Hacer estas prácticas te van a dar la
oportunidad de poder aplicar lo aprendido en la teoría, entender mejor los conceptos y probar tus habilidades.

La recomendación de la cátedra es que:

\begin{itemize}
\item \textbf{Leas MUY bien el enunciado}: Analices el eneunciado y entiendas exactamente lo que hay que hacer. Parece
sencillo, pero es común resolver un ejercicio diferente al planteado.

\item \textbf{No busques soluciones óptimas inmediatamente}: Primero intentá resolver el problema y luego pensá si la
solución es óptima.

\item \textbf{Dediques tiempo a pensar}: No te desesperes si no se te ocurre la solución inmediatamente. Es común que
las soluciones no salgan a la primera. Pensá en el problema, quizás volvé a leer la teória y fijate si se te ocurre. Son
problemas complejos y a veces hay que darles tiempo para que se asienten.

\item \textbf{¡No busques soluciones rápidas!}: No busques rápido en internet la solución o vayas a leer la solución a
la guía de resoluciones inmediátamente. Cada solución que leas rápido te va a dar la falsa sensación de comprensión y te
va a sacar la posibilidad de tener el momento \textbf{¡AHA!} donde realmente entendiste cómo resolver un problema.
\textbf{¡Te entendemos, es difícil a veces!}, pero es parte del proceso de aprendizaje.

\item \textbf{¡No te desanimes!}: Si volviste a pensarlo un tiempo y no se te ocurre nada, es momento de dejar el
problema por un tiempo y retomarlo luego.

\item \textbf{¡Volví al problema luego de un tiempo y no me sale!}: Si volviste a pensar el problema y no se te ocurre
nada, es momento de leer la solución. No te sientas mal por esto, pero cuando te sientes a leer la solución. El proceso
de leer la solución implica \textit{entenderla}, y NO copiarla. Una vez que entiendas la solución, esperá un tiempo para
escribrla y probarla.

\item \textbf{¿Y si no entiendo la solución?}: Si no entendiste la solución, anotá las dudas, tratá de pensar qué es
lo que te falta y preguntá a los docentes de la cátedra. ¡Nunca te quedes con la duda!

\item \textbf{ChatGPT (cualquier LLM) lo resuelve todo}: ¡Es verdad!, pero como cualquier herramienta, cuanta más teoría
  sepamos, mejor podremos utilizarla.

\textbf{¡Suerte en la práctica!}
\end{itemize}
\end{tcolorbox}
