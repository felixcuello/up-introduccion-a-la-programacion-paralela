% \documentclass[10pt, twocolumn, a4paper]{article}
\documentclass[12pt,a4paper]{article}

\usepackage[backend=biber, style=ieee]{biblatex}                        % To include the bibliography
\usepackage[left=2cm, right=2cm, top=2.5cm, bottom=2.5cm]{geometry}     % To set the margins
\usepackage[noend]{algpseudocode}
\usepackage[table]{xcolor}                                              % For coloring cells

\usepackage{algorithm}                                                  % To include algorithms
\usepackage{amsfonts}                                                   % To include math fonts:ToggleTerm direction=float
\usepackage{amsmath}                                                    % To include Mathematic symbols
\usepackage{authblk}                                                    % To format author affiliations
\usepackage{caption}                                                    % For caption spacing
\usepackage{enumitem}                                                   % To customize lists (items like i, ii, iii, iv)
\usepackage{float}                                                      % To place figures where you want them
\usepackage{graphicx}                                                   % To include images
\usepackage{hyperref}                                                   % To include hyperlinks
\usepackage{lipsum}                                                     % TODO: remove this
\usepackage{listings}                                                   % To include code
\usepackage{tabularx}                                                   % For equal-width columns
\usepackage{tcolorbox}                                                  % To make colored boxes
\usepackage{tikz}                                                       % To draw graphs
\usepackage{titlesec}                                                   % To format section titles
\usepackage{xcolor}                                                     % To define colors

\usetikzlibrary{graphs,graphs.standard}
\usetikzlibrary{positioning}

\addbibresource{./references.bib}

% Esto es para poder hacer cajitas de código con el fondo gris
\lstset{
    language=C++,
    basicstyle=\ttfamily\footnotesize,
    keywordstyle=\color{blue}\bfseries,
    stringstyle=\color{green!60!black},
    commentstyle=\color{gray},
    backgroundcolor=\color{gray!05},
    frame=single,
    numbers=left,
    numberstyle=\footnotesize,
    stepnumber=1,
    numbersep=10pt,
    tabsize=2,
    showstringspaces=false,
    captionpos=b,
    breaklines=true,
}

% Para poder hacer flechas
\usetikzlibrary{shapes, arrows}

% Sección de definiciones
\titleformat{\section}{\Large\bfseries}{\thesection}{1em}{}
\titleformat{\subsection}{\large\bfseries}{\thesubsection}{1em}{}

% Caja de colores
\definecolor{mint}{RGB}{202,251,202}
\definecolor{yellow}{RGB}{255,255,202}
\definecolor{red}{RGB}{255,202,202}

% Variables globales para el documento
\newcommand{\currentsemester}{Segundo Semestre}
\newcommand{\currentyear}{2025}

% Esto es para poder agregar comentarios al código
\newcommand{\comentario}[1]{\textcolor{gray}{// #1}}
