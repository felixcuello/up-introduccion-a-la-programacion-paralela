\begin{tcolorbox}[colback=red,colframe=red!75!black,arc=0pt,outer arc=0pt]
  \textbf{¡STOP!} \\

Estas son las ¡SOLUCIONES! a los problemas \textbf{¿estás seguro que vas a revisar esto?} \\
La idea de estas soluciones es que estén aquí para que puedas revisarlas cuando ya agotaste todas las posibilidades de 
poder resolver el problema por tu cuenta. De todas formas te recomendamos algunas cosas:

\begin{itemize}
\item \textbf{NO las leas sin intentarlo}: Porque si no lo intentaste y lees la solución, muchas veces vas a intentar
  llegar a la solución escrita y no intentar solucionar el problema. Puede parecer lo mismo, pero no lo es. La idea de
  la práctica es llegar a estas soluciones por tu cuenta.

\item \textbf{¡Pero no me salen!}: El momento de real aprendizaje ocurre cuando intentás suficiente y, de repente, te
  sale. Es un momento de \textbf{¡AHA!} donde entendiste cómo resolver el problema. Siempre intentá un poquito más antes
  de leer las soluciones.

\item \textbf{¡Ya lo pensé de varias formas diferentes!... ¡no me sale!}: Parte del aprendizaje es intentarlo. Si creés
  que lo intentaste suficiente y aún así no te salió, es el momento de leer la solución. Cada una de estas soluciones
  tiene alguna pista antes de la respuesta, intentá leer las pistas y ver si con eso te sale. Las pistas van de más
  difícil a más fácil, para que tengas un camino de aprendizaje.

\item \textbf{¡Leí todas las pistas y no entiendo!}: No te preocupes, a veces pasa. Es momento de leer la solución,
  intentar pensar cómo llegar a esa solución y sino preguntar al docente.

\end{itemize}
\end{tcolorbox}
