% \documentclass[10pt, twocolumn, a4paper]{article}
\documentclass[12pt,a4paper]{article}

\usepackage[backend=biber, style=ieee]{biblatex}                        % To include the bibliography
\usepackage[left=2cm, right=2cm, top=2.5cm, bottom=2.5cm]{geometry}     % To set the margins
\usepackage[noend]{algpseudocode}
\usepackage[table]{xcolor}                                              % For coloring cells

\usepackage{algorithm}                                                  % To include algorithms
\usepackage{amsfonts}                                                   % To include math fonts:ToggleTerm direction=float
\usepackage{amsmath}                                                    % To include Mathematic symbols
\usepackage{authblk}                                                    % To format author affiliations
\usepackage{caption}                                                    % For caption spacing
\usepackage{float}                                                      % To place figures where you want them
\usepackage{graphicx}                                                   % To include images
\usepackage{hyperref}                                                   % To include hyperlinks
\usepackage{lipsum}                                                     % TODO: remove this
\usepackage{listings}                                                   % To include code
\usepackage{tabularx}                                                   % For equal-width columns
\usepackage{tcolorbox}                                                  % To make colored boxes
\usepackage{tikz}                                                       % To draw graphs
\usepackage{titlesec}                                                   % To format section titles
\usepackage{xcolor}                                                     % To define colors

\usetikzlibrary{graphs,graphs.standard}
\usetikzlibrary{positioning}

\addbibresource{./references.bib}

% Esto es para poder hacer cajitas de código con el fondo gris
\lstset{
    language=C++,
    basicstyle=\ttfamily\small,
    keywordstyle=\color{blue}\bfseries,
    stringstyle=\color{green!60!black},
    commentstyle=\color{gray},
    backgroundcolor=\color{gray!05},
    frame=single,
    numbers=left,
    numberstyle=\small,
    stepnumber=1,
    numbersep=10pt,
    tabsize=2,
    showstringspaces=false,
    captionpos=b,
}

% Para poder hacer flechas
\usetikzlibrary{shapes, arrows}

% Sección de definiciones
\titleformat{\section}{\Large\bfseries}{\thesection}{1em}{}
\titleformat{\subsection}{\large\bfseries}{\thesubsection}{1em}{}

% Caja de colores
\definecolor{mint}{RGB}{202,251,202}
\definecolor{yellow}{RGB}{255,255,202}
\definecolor{red}{RGB}{255,202,202}

% Variables globales
\newcommand{\currentsemester}{Segundo Semestre}
\newcommand{\currentyear}{2025}


\begin{document}

\begin{center}
  \LARGE\textbf{Introducción a la Programación Paralela (CUDA)} \\
  \Large{Práctica 01 - Introducción} \\
  \normalsize{\currentsemester, \currentyear} \\
  \vspace{1em}
  \hrule
\end{center}

\setcounter{section}{1}

\begin{tcolorbox}[colback=yellow,colframe=red!75!black,arc=0pt,outer arc=0pt]
  \textbf{ATENCIÓN CON LAS PRÁCTICAS} \\

En programación, la práctica es un componente fundamental del aprendizaje. Hacer estas prácticas te van a dar la
oportunidad de poder aplicar lo aprendido en la teoría, entender mejor los conceptos y probar tus habilidades.

La recomendación de la cátedra es que:

\begin{itemize}
\item \textbf{Leas MUY bien el enunciado}: Analices el eneunciado y entiendas exactamente lo que hay que hacer. Parece
sencillo, pero es común resolver un ejercicio diferente al planteado.

\item \textbf{No busques soluciones óptimas inmediatamente}: Primero intentá resolver el problema y luego pensá si la
solución es óptima.

\item \textbf{Dediques tiempo a pensar}: No te desesperes si no se te ocurre la solución inmediatamente. Es común que
las soluciones no salgan a la primera. Pensá en el problema, quizás volvé a leer la teória y fijate si se te ocurre. Son
problemas complejos y a veces hay que darles tiempo para que se asienten.

\item \textbf{¡No busques soluciones rápidas!}: No busques rápido en internet la solución o vayas a leer la solución a
la guía de resoluciones inmediátamente. Cada solución que leas rápido te va a dar la falsa sensación de comprensión y te
va a sacar la posibilidad de tener el momento \textbf{¡AHA!} donde realmente entendiste cómo resolver un problema.
\textbf{¡Te entendemos, es difícil a veces!}, pero es parte del proceso de aprendizaje.

\item \textbf{¡No te desanimes!}: Si volviste a pensarlo un tiempo y no se te ocurre nada, es momento de dejar el
problema por un tiempo y retomarlo luego.

\item \textbf{¡Volví al problema luego de un tiempo y no me sale!}: Si volviste a pensar el problema y no se te ocurre
nada, es momento de leer la solución. No te sientas mal por esto, pero cuando te sientes a leer la solución. El proceso
de leer la solución implica \textit{entenderla}, y NO copiarla. Una vez que entiendas la solución, esperá un tiempo para
escribrla y probarla.

\item \textbf{¿Y si no entiendo la solución?}: Si no entendiste la solución, anotá las dudas, tratá de pensar qué es
lo que te falta y preguntá a los docentes de la cátedra. ¡Nunca te quedes con la duda!

\item \textbf{ChatGPT (cualquier LLM) lo resuelve todo}: ¡Es verdad!, pero como cualquier herramienta, cuanta más teoría
  sepamos, mejor podremos utilizarla.

\textbf{¡Suerte en la práctica!}
\end{itemize}
\end{tcolorbox}


\newpage

\tableofcontents

\newpage

\subsection{¿Qué es escalabilidad vertical? ¿por qué tiene un límite?}

\subsection{¿Qué es la escalbilidad horizontal? ¿Cómo mejora la escalabilidad vertical?}

\subsection{¿Por qué es importante medir la complejidad algorítmica?}

\subsection{¿Qué es la complejidad en espacio?}

\subsection{¿Qué es la complejidad en tiempo?}

\subsection{¿Por qué no se utilizan GPUs y programación paralela para todo?}

\subsection{¿Qué es la Ley de Amdahl?}

\subsection{¿A qué se denomina \textit{Nick's Class}? ¿por qué es importante en la programación paralela?}

\newpage

\vspace{1em}

\printbibliography


\end{document}
