% \documentclass[10pt, twocolumn, a4paper]{article}
\documentclass[12pt,a4paper]{article}

\usepackage[backend=biber, style=ieee]{biblatex}                        % To include the bibliography
\usepackage[left=2cm, right=2cm, top=2.5cm, bottom=2.5cm]{geometry}     % To set the margins
\usepackage[noend]{algpseudocode}
\usepackage[table]{xcolor}                                              % For coloring cells

\usepackage{algorithm}                                                  % To include algorithms
\usepackage{amsfonts}                                                   % To include math fonts:ToggleTerm direction=float
\usepackage{amsmath}                                                    % To include Mathematic symbols
\usepackage{authblk}                                                    % To format author affiliations
\usepackage{caption}                                                    % For caption spacing
\usepackage{float}                                                      % To place figures where you want them
\usepackage{graphicx}                                                   % To include images
\usepackage{hyperref}                                                   % To include hyperlinks
\usepackage{lipsum}                                                     % TODO: remove this
\usepackage{listings}                                                   % To include code
\usepackage{tabularx}                                                   % For equal-width columns
\usepackage{tcolorbox}                                                  % To make colored boxes
\usepackage{tikz}                                                       % To draw graphs
\usepackage{titlesec}                                                   % To format section titles
\usepackage{xcolor}                                                     % To define colors

\usetikzlibrary{graphs,graphs.standard}
\usetikzlibrary{positioning}

\addbibresource{./references.bib}

% Esto es para poder hacer cajitas de código con el fondo gris
\lstset{
    language=C++,
    basicstyle=\ttfamily\small,
    keywordstyle=\color{blue}\bfseries,
    stringstyle=\color{green!60!black},
    commentstyle=\color{gray},
    backgroundcolor=\color{gray!05},
    frame=single,
    numbers=left,
    numberstyle=\small,
    stepnumber=1,
    numbersep=10pt,
    tabsize=2,
    showstringspaces=false,
    captionpos=b,
}

% Para poder hacer flechas
\usetikzlibrary{shapes, arrows}

% Sección de definiciones
\titleformat{\section}{\Large\bfseries}{\thesection}{1em}{}
\titleformat{\subsection}{\large\bfseries}{\thesubsection}{1em}{}

% Caja de colores
\definecolor{mint}{RGB}{202,251,202}
\definecolor{yellow}{RGB}{255,255,202}
\definecolor{red}{RGB}{255,202,202}

% Variables globales
\newcommand{\currentsemester}{Segundo Semestre}
\newcommand{\currentyear}{2025}


\begin{document}

\begin{center}
    \LARGE\textbf{Programación Paralela} \\
    \Large{Apéndice: C++} \\
    \normalsize{\currentsemester, \currentyear} \\
    \vspace{1em}
    \hrule
\end{center}

\section{Introducción a Docker}

Docker es una plataforma de software que permite a los desarrolladores empaquetar, enviar y ejecutar aplicaciones en
contenedores. En su esencia, Docker encapsula una aplicación junto con todas sus dependencias dentro de un contenedor,
un entorno de ejecución aislado que comparte el kernel del sistema operativo host lo cual lo hace mucho más ligero y
rápido que distingue a Docker de las máquinas virtuales tradicionales, las cuales virtualizan el hardware y, por ende,
conllevan una sobrecarga significativa en términos de recursos.

Docker se compone de:

\begin{itemize}
  \item \textbf{Docker Engine}: Es el componente central que gestiona la creación y ejecución de contenedores. Incluye el
    Docker Daemon (\texttt{dockerd}), que escucha las solicitudes de la API de Docker y gestiona los objetos de Docker
    (imágenes, contenedores, redes, volúmenes). También incluye la CLI de Docker (\texttt{docker}), que permite a los
    usuarios interactuar con el Docker Daemon.

  \item \textbf{Imágenes Docker}: \textit{Templates} de solo lectura con instrucciones para crear un contenedor. Se
    construyen a partir de un Dockerfile, un archivo de texto con instrucciones paso a paso. Las imágenes se almacenan en
    registros de imágenes, como Docker Hub.

  \item \textbf{Contenedores Docker}: Instancias en ejecución de imágenes Docker. Proporcionan un entorno aislado para
    ejecutar aplicaciones. Pueden iniciarse, detenerse, eliminarse y gestionarse mediante la CLI de Docker.

  \item \textbf{\texttt{Dockerfile}}: Un archivo de texto que contiene las instrucciones para construir una imagen Docker.
    Define la imagen base, las dependencias, la configuración y los comandos para ejecutar la aplicación.
\end{itemize}

\section{Docker en la materia}

Dado que no todos vamos a querer instalar el compilador de nvidia \texttt{nvcc} en nuestras computadoras, e incluso, hay
algunos sistemas operativos donde puede ser tedioso, será más fácil utilizar directamente un contenedor de Docker que
nos permita tener a todos el mismo entorno de desarrollo para trabajar.

Para comenzar te recomiendo comenzar leyendo el archivo
\href{https://github.com/felixcuello/up-materias/blob/main/programacion_paralela/docker/README.md}{\texttt{README.md}}
de este repositorio para ver cómo instalar Docker y cómo utilizar el contenedor preparado para la materia, que ya tiene
el compilador \texttt{nvcc} instalado.

En el archivo \texttt{README.md} se explica cómo crear e ingresar al contenedor utilizando \texttt{make}, si no tienen
el \texttt{make} instalado en su computadora, pueden abrir el archivo \texttt{Makefile} que contiene las acciones que se
pueden ejecutar y pueden correr los comandos de \texttt{docker} manualmente.

\newpage

\vspace{1em}

\printbibliography


\end{document}
